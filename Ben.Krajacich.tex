%!TEX TS-program = xelatex
%!TEX encoding = UTF-8 Unicode
% Awesome CV LaTeX Template for CV/Resume
%
% This template has been downloaded from:
% https://github.com/posquit0/Awesome-CV
%
% Author:
% Claud D. Park <posquit0.bj@gmail.com>
% http://www.posquit0.com
%
%
% Adapted to be an Rmarkdown template by Mitchell O'Hara-Wild
% 23 November 2018
%
% Template license:
% CC BY-SA 4.0 (https://creativecommons.org/licenses/by-sa/4.0/)
%
%-------------------------------------------------------------------------------
% CONFIGURATIONS
%-------------------------------------------------------------------------------
% A4 paper size by default, use 'letterpaper' for US letter
\documentclass[11pt, a4paper]{awesome-cv}

% Configure page margins with geometry
\geometry{left=1.4cm, top=.8cm, right=1.4cm, bottom=1.8cm, footskip=.5cm}

% Specify the location of the included fonts
\fontdir[fonts/]

% Color for highlights
% Awesome Colors: awesome-emerald, awesome-skyblue, awesome-red, awesome-pink, awesome-orange
%                 awesome-nephritis, awesome-concrete, awesome-darknight

\definecolor{awesome}{HTML}{414141}

% Colors for text
% Uncomment if you would like to specify your own color
% \definecolor{darktext}{HTML}{414141}
% \definecolor{text}{HTML}{333333}
% \definecolor{graytext}{HTML}{5D5D5D}
% \definecolor{lighttext}{HTML}{999999}

% Set false if you don't want to highlight section with awesome color
\setbool{acvSectionColorHighlight}{true}

% If you would like to change the social information separator from a pipe (|) to something else
\renewcommand{\acvHeaderSocialSep}{\quad\textbar\quad}

\def\endfirstpage{\newpage}

%-------------------------------------------------------------------------------
%	PERSONAL INFORMATION
%	Comment any of the lines below if they are not required
%-------------------------------------------------------------------------------
% Available options: circle|rectangle,edge/noedge,left/right

\name{Ben}{Krajacich}

\position{Post-doctoral Researcher}
\address{Rockville, MD}

\mobile{406 799 8383}
\email{\href{mailto:bkrajacich@gmail.com}{\nolinkurl{bkrajacich@gmail.com}}}
\homepage{bkraj.netlify.com}
\github{benkraj}
\twitter{bkraj}

% \gitlab{gitlab-id}
% \stackoverflow{SO-id}{SO-name}
% \skype{skype-id}
% \reddit{reddit-id}


\usepackage{booktabs}

% Templates for detailed entries
% Arguments: what when with where why
\usepackage{etoolbox}
\def\detaileditem#1#2#3#4#5{%
\cventry{#1}{#3}{#4}{#2}{\ifx#5\empty\else{\begin{cvitems}#5\end{cvitems}}\fi}\ifx#5\empty{\vspace{-4.0mm}}\else\fi}
\def\detailedsection#1{\begin{cventries}#1\end{cventries}}

% Templates for brief entries
% Arguments: what when with
\def\briefitem#1#2#3{\cvhonor{}{#1}{#3}{#2}}
\def\briefsection#1{\begin{cvhonors}#1\end{cvhonors}}

\providecommand{\tightlist}{%
	\setlength{\itemsep}{0pt}\setlength{\parskip}{0pt}}

%------------------------------------------------------------------------------



\begin{document}

% Print the header with above personal informations
% Give optional argument to change alignment(C: center, L: left, R: right)
\makecvheader

% Print the footer with 3 arguments(<left>, <center>, <right>)
% Leave any of these blank if they are not needed
% 2019-02-14 Chris Umphlett - add flexibility to the document name in footer, rather than have it be static Curriculum Vitae
\makecvfooter
  {May 2020}
    {Ben Krajacich~~~·~~~Curriculum Vitae}
  {\thepage}


%-------------------------------------------------------------------------------
%	CV/RESUME CONTENT
%	Each section is imported separately, open each file in turn to modify content
%------------------------------------------------------------------------------



\hypertarget{biography}{%
\section{Biography}\label{biography}}

I am a vector ecologist and microbiologist who is broadly interested in
medical entomology that never strays too far from natural systems. I
have always had a focus on vector-borne disease, with my post-bac on
tick-borne relapsing fever in soft ticks, my phd work was on the
development of tools to study the use of ivermectin as a malaria control
measure in West Africa (Senegal, Liberia, and Burkina Faso), and my
post-doc work focuses on the dry-season persistence mechanisms of
Anopheles spp. mosquitoes in Mali.

\hypertarget{education}{%
\section{Education}\label{education}}

\detailedsection{\detaileditem{Ph.D. -  Advisor: Dr. Brian D. Foy  - Tools and Techniques for the Study and Evaluation of Malaria Control Measures in West Africa }{2011 - 2016}{Colorado State University}{Fort Collins, CO}{\empty}\detaileditem{BS - Advisor: Dr. Edward Dratz - Seeking a Novel Screen for Human Disease: A Foundation for Characterization of Albumin-Bound Lipids and Other Compounds in Human Blood Plasma}{2005 - 2009}{Montana State University}{Bozeman, MT}{\empty}}

\hypertarget{employment}{%
\section{Employment}\label{employment}}

\detailedsection{\detaileditem{Malaria Research Program Postdoctoral Fellow - Advisor: Dr. Tovi Lehmann - Dry season persistence mechanisms of Anopheles coluzzii in Mali}{2016 - Present}{National Institutes of Health}{Rockville, MD}{\empty}\detaileditem{Graduate Teaching Assistant - Immunology Laboratory, Parasitology and Vector Biology}{2011 - 2012}{Colorado State University}{Fort Collins, CO}{\empty}\detaileditem{Post-Baccalaureate IRTA - Advisor: Dr. Tom Schwan - Development of a vaccine candidate against the Tick-Borne Relapsing Fever Agent Borrelia Hermsii  }{2009 - 2011}{National Institutes of Health}{Hamilton, MT}{\empty}}

\hypertarget{summary-of-skills}{%
\section{Summary of Skills}\label{summary-of-skills}}

\begin{itemize}
\tightlist
\item
  Domestic and international field experience including work in French
  and English-speaking West Africa. Involved entomological surveillance
  (aspiration catches, human landing catch, CDC light traps, larval
  collection, tick dragging), taxonomic identification, work with human
  volunteers, analysis and collection of human blood samples, and
  involvement in clinical trials (Ivermectin MDA as an anti-malarial
  endectocide).
\item
  Extensive wet lab experience including PCR, RT-PCR, QT-NASBA, droplet
  digital PCR, next-generation sequencing, molecular cloning, generation
  of expression vectors, and generation of transgenic cell lines,
  including novel assay development for these approaches. Immunological
  techniques including western blotting, ELISA, immunohistochemistry,
  and vaccine production.
\item
  Fluency in R, with extensive experiences in data
  manipulation/wrangling of complex data sets, GIS/mapping, analysis of
  next-generation sequencing data sets (16S microbiome/RNA-seq/WGS),
  machine learning, dashboard development for field projects with
  Shiny/REDcap, extraction of weather data for climatic data.
\item
  Insect vector husbandry (rearing, maintenance, feeding experiments)
  including \emph{Ornithodoros} soft ticks, and \emph{Anopheles},
  \emph{Culex}, \emph{Aedes} mosquitoes.
\item
  Training and experience in the use and handling of experimental
  laboratory mice, and a variety of wild animals. Experience in the
  collection of blood samples through various sites for serological
  testing.
\end{itemize}

\hypertarget{publications}{%
\section{Publications}\label{publications}}

\begin{itemize}
\item
  Huestis DL, Dao A, Diallo M, Sanogo ZL, Samake D, Yaro AS, Ousman Y,
  Linton Y-M, Krishna A, Veru L, \textbf{Krajacich BJ}, Faiman R, Florio
  J, Chapman JW, Reynolds DR, Weetman D, Mitchell R, Donnelly MJ,
  Talamas E, Chamorro L, Strobach E and Lehmann T. Windborne
  long-distance migration of malaria mosquitoes in the Sahel. Nature. In
  press. 2019.
\item
  Faiman R, Dao A, Yaro AS, Diallo M, Djibril S, Sonogo ZL, Ousmane Y,
  Sullivan M, Veru L, \textbf{Krajacich BJ}, Krishna A, Matthews J,
  France CAM, Hamer G, Hobson KA, Lehmann T. Marking mosquitoes in their
  natural larval sites using 2H-enriched water: a promising approach for
  tracking over extended temporal and spatial scales. Methods in Ecology
  and Evolution. 2019. \url{doi:10.1111/2041-210x.13210}
\item
  \textbf{Krajacich BJ}, Huestis DL, Dao A, Yaro AS, Diallo M, Krishna
  A, Xu J, Lehmann T. Investigation of the Seasonal Microbiome of
  Anopheles coluzzii in Mali. PLOS ONE. 2018. 13(3): e0194899
\item
  \textbf{Krajacich BJ}, Meyers JI, Alout H, Dabiré KR, Dowell FE, Foy
  BD. Validation of Near Infrared Spectroscopy for age-grading of wild
  Anopheles gambiae. Parasites and Vectors. 2017. 10:552
\item
  Fauver, JR, Grubaugh ND, \textbf{Krajacich BJ}, Weger J, Fakoli LS,
  Bolay F, Diclaro J, Dabiré KR, Foy BD, Brackney D, Ebel GD, Stenglein
  M. West African Anopheles gambiae mosquitoes harbor a taxonomically
  diverse virome including new insect-specific flaviviruses,
  mononegaviruses, and totiviruses. 2016. 498:288-299
\item
  \textbf{Krajacich BJ}, Lopez JE, Raffel SJ, Schwan TG. (2015).
  Vaccination with the variable tick protein of the relapsing fever
  spirochete Borrelia hermsii protects mice from infection by tick-bite.
  Parasites and Vectors. 2015;8(546).
  \url{doi:10.1186/s13071-015-1170-1}.
\item
  Grubaugh ND, Sharma S, \textbf{Krajacich BJ}, Fakoli LS, Bolay FK,
  DiClaro JW, Johnson WE, Ebel GD, Foy BD, Brackney DE. (2015).
  Xenosurveillance: a novel mosquito-based approach for examining the
  human-pathogen landscape. PLoS Negl Trop Dis. 2015;9(3):e0003628.
  \url{doi:10.1371/journal.pntd.0003628}.
\item
  \textbf{Krajacich BJ}, Slade JR, Mulligan RF, LaBrecque B, Alout H,
  Grubaugh ND, Meyers JI, Fakoli LS, Bolay FK, Brackney DE, Burton T A.,
  Seaman J A., Diclaro JW, Dabire RK, Foy BD. (2015). Sampling
  Host-Seeking Anthropophilic Mosquito Vectors in West Africa:
  Comparisons of an Active Human-Baited Tent-Trap Against Gold Standard
  Methods. Am J Trop Med Hyg. 2015;92(2):415--421.
  \url{doi:10.4269/ajtmh.14-0303}.
\item
  \textbf{Krajacich BJ}, Slade J.R., Mulligan R.F., LaBrecque B.,
  Kobylinski K.C., Gray M., Kuklinski W.S., Burton T.A., Seaman J.A.,
  Sylla M., Foy B.D. (2014). Design and Testing of a Novel, Protective
  Human-Baited Tent Trap for the Collection of Anthropophilic Disease
  Vectors. Journal of Medical Entomology. 51(1):253-263.
\item
  Lopez J.E., McCoy B.N, \textbf{Krajacich BJ}, Schwan T.G. (2011).
  Acquisition and subsequent transmission of Borrelia hermsii by the
  soft tick, Ornithodoros hermsi. Journal of Medical Entomology.
  48(4):891-895.
\end{itemize}

\hypertarget{presentations}{%
\section{Presentations}\label{presentations}}

\begin{itemize}
\item
  \textbf{Krajacich BJ}, Graber L, Faiman R, Sullivan M, Lehmann T.
  Extension of lifespan in Anopheles coluzzii mosquitoes by climatic
  modulation. American Society for Tropical Medicine and Hygiene
  Conference -- 2018 -- New Orleans, LA -- Oral Presentation
\item
  \textbf{Krajacich BJ}, Huestis DL, Dao A, Yaro AS, Diallo M, Krishna
  A, Xu J, Lehmann T. Investigation of the Seasonal Microbiome of
  Anopheles coluzzii in Mali. Entomological Society of America
  conference, 2017, Denver, CO.
\item
  \textbf{Krajacich BJ}, Meyers J.I., Alout H., Dabiré R.K., Dowell
  F.E., Foy B.D., Validation of Near Infrared Spectroscopy for the
  age-grading of wild Anopheles gambiae. Oral Presentation at the
  JHMRI's ``The Future of Malaria Research'' conference 2016, Rockville,
  MD.
\item
  \textbf{Krajacich BJ}, Molina-Cruz A., Barillas-Mury C., Foy, B.D.,
  Use of mosquito bloodmeals as epidemiological tools to study malaria
  transmission. Oral Presentation at 2016 CVMBS Research Day.
\item
  \textbf{Krajacich BJ}, Molina-Cruz A., Grubaugh N.D., Brackney D.E.,
  Alout H., Meyers J.I., Fakoli L.S., Bolay F.K., DiClaro J.W., Dabiré
  R.K., Barillas-Mury C., Foy, B.D., Development and utilization of
  molecular methods for the detection of Plasmodium falciparum in
  mosquito bloodmeals. 2015 Keystone Symposia Meeting -- The Arthropod
  Vector: The Controller of Transmission.
\item
  \textbf{Krajacich BJ}, Grubaugh ND, Brackney DE, Alout H, Meyers JI,
  Fakoli LS, Bolay FK, DiClaro JW, Dabiré RK, Foy BD. Detection of
  Plasmodium falciparum in the Bloodmeal of Anopheles gambiae using
  Quantitative Nucleic Acid Sequence Based Amplification (QT-NASBA).
  American Society for Tropical Medicine and Hygiene 63nd Annual Meeting
  2014.
\item
  \textbf{Krajacich BJ}, Slade J.R., Mulligan R.F., LaBrecque B.,
  Kobylinski K.C., Gray M., Sylla M., Burton T.A., Kuklinski W.S.,
  Seaman J.A., DiClaro J.W. II, Fakoli L.S. III, Dabiré R.K., Bolay
  F.K., Foy B.D. Demonstration and Analysis of a Safe, Novel,
  Human-baited Tent Trap for the Collection of Anthropophagic Disease
  Vectors. American Society for Tropical Medicine and Hygiene 62nd
  Annual Meeting 2013.
\item
  \textbf{Krajacich BJ}, Slade J.R., Kobylinski K.C., Gray M., Burton
  T.A., Kuklinski W.S., Seaman J.A., Sylla M., Foy B.D. Demonstration
  and Analysis of a Safe, Novel, Human-baited Tent Trap for the
  Collection of Anthropophagic Disease Vectors. 2013 CMB/MCIN/BMB/MIP
  Spring Poster Symposium.
\item
  \textbf{Krajacich BJ}, Bowden J.N., Gillespie, G.D.,Dratz, E.A.,
  Seeking a Novel Screen for Disease: Fluorescence Lifetime Monitoring
  of Plasma Thermal Denaturation. 2008 MSU Student Research Celebration.
  \textbf{Krajacich BJ}, Slade J.R., Kobylinski K.C., Gray M., Burton
  T.A., Kuklinski W.S., Seaman J.A., Sylla M., Foy B.D. Demonstration
  and Analysis of a Safe, Novel, Human-baited Tent Trap for the
  Collection of Anthropophagic Disease Vectors. 2013 CVMBS Research Day.
\item
  \textbf{Krajacich BJ}, Bowden J.N., Dratz, E.A., 2009. Seeking a Novel
  Screen for Human Disease: A Foundation for Characterization of
  Albumin-Bound Lipids and Other Compounds in Human Blood Plasma. 2009
  MSU Student Research Celebration.
\end{itemize}

\hypertarget{professional-memberships}{%
\section{Professional Memberships}\label{professional-memberships}}

\begin{itemize}
\item
  American Society for Tropical Medicine and Hygiene (ASTMH) -- since
  2011
\item
  American Committee of Medical Entomology (ACME) -- since 2011
\end{itemize}

\hypertarget{community-outreach-and-service}{%
\section{Community Outreach and
Service}\label{community-outreach-and-service}}

\begin{itemize}
\item
  \textbf{Journal Referee}, Parasites and Vectors, Nature: Scientific
  Reports, Malaria Journal, PeerJ, PLOS Biology, PLOS ONE
\item
  \textbf{Guest Community Scientist}, HB Woodlawn Secondary Program,
  Arlington, VA
\end{itemize}

\hypertarget{awards-and-grants}{%
\section{Awards and Grants}\label{awards-and-grants}}

\begin{itemize}
\tightlist
\item
  Summer 2018 -- Awarded Travel Bursary to attend Wellcome Trust /
  Sanger Institute Genomic Epidemiology of Malaria conference in
  Hinxton, UK
\item
  Supplemental Funding Request -- FY2017 -- National Institutes of
  Health - ``Acquisition of the Thermo-Scientific KingFisher Flex Robot
  for High throughput DNA/RNA extraction'' - \$55,312.50
\item
  Fall 2016-Fall 2019 -- Awarded Malaria Research Program Postdoctoral
  Fellowship
\item
  Fall 2014 -- Department of Microbiology, Immunology, and Pathology
  Travel Grant to attend the American Society for Tropical Medicine and
  Hygiene annual meeting.
\item
  Fall 2009 -- Recipient of National Institutes of Health
  Post-Baccalaureate Intramural Research Training Award
\item
  Fall 2008 -- IdeA Network for Biomedical Research Excellence (INBRE)
  Program Grant
\item
  Summer 2008 -- Montana State University's University Scholar's Program
  Funding Grant
\item
  Spring 2008 -- IdeA Network for Biomedical Research Excellence (INBRE)
  Program Grant
\end{itemize}

\end{document}
